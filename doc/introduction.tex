\section{Introduction}

During my 2009 SURF project (an overview of which is given in (REF)),
I constructed a prototype implemention of the spectroscopic-mode
data reduction pipeline (DRP) for the MOSFIRE near-infrared imaging
spectrometer. A closely-related simulator for spectroscopic-mode MOSFIRE
observations was also constructed. This report describes the algorithms
used in these programs.

\section{Instrument Overview}

In order to derive information about astronomical objects
from the instrument data, we need some description of how light from these
objects, among other factors, affects the instrument. Such a description
is also necessary for going in the other direction, i.e.\ simulating
the output of the instrument.

As described in (REF), the behaviour of the instrument can (for the
purposes of the DRP) be summarised by a series of maps. The first of
these is the `sky image' (spectral radiance) map,
\[
L_{sky} = L_{sky} (\lambda, x, y)
\]
where $x,y$ are coordinates determining position in the sky (e.g. RA
and DEC), $\lambda$ the wavlength, and $L_{sky}$ the incident power
per unit projected receiving area per unit solid angle. That is,
$L_{sky}(\lambda, x, y)\, dA \, d\lambda$, evaluated for a specific
point in space, gives the power received from the direction $(x,y)$
by a small tile of area area $dA$ positioned at that point (with
normal pointed towards $(x,y)$), in the frequency range $\lambda \pm
d\lambda/2$. In general, $L_{sky}$ will depend on your position in
space --- however, to a good approximation it will be constant across
the surface of the telescope's mirror.

As the telescope is within the Earth's atmosphere, the $L_{sky}$
we observe is affected by the atmosphere, with
\[
L_{sky}(\lambda_0, x_0, y_0) = L_{atm}(\lambda_0, x_0, y_0) + \int d\lambda\, dx\, dy\, L_{space}(\lambda, x, y) t_{atm}[\lambda, x, y](\lambda_0, x_0, y_0)
\]
where $L_{atm}$ is due to radiation emitted by the atmosphere, and
$t_{atm}$ is the atmosphere's transfer function. In general, this will
have the form $t_{atm}[\lambda, x, y] = \epsilon_{atm}(\lambda, x,
y) P[\lambda, x, y]$, where $\epsilon_{atm} (\lambda, x, y) \in
\mathbb{R}^+$ is the atmosphere's transfer efficiency, and $P[\lambda,
x, y]$ the normalised point spread function (PSF), which depends on the
seeing conditions.

$L_{sky}$ summarises what the telescope's mirror sees, but we're
actually interested in what the instrument's detector sees. Since
the dectector is, to a good approximation, equally sensitive to all
frequencies in the relevent bands, this is summarised by the `detector
image' (irradiance) map,
\[
E_{det} = E_{det} (p_x, p_y, L_{sky})
\]
where $p_x, p_y$ are coordinates for the detector surface, and
$E_{det}$ is the incident power per unit area at this surface
(integrated over the appropriate frequency range).

Since the telescope and instrument form a linear optical system,
we should have
\[
E_{det}(p_x, p_y, L_{sky}) = \int d\lambda\, dx\, dy\, L_{sky}(\lambda, x, y) t_{inst}[\lambda, x, y](p_x, p_y)
\]
where $t_{inst}$ is the combined transfer function of the telescope and the instrument.
This splits as 
\[
t_{inst}[\lambda,x, y](p_x, p_y) = \epsilon_{inst} (\lambda, x, y)
Q[\lambda, x, y]((p_x, p_y) - g (\lambda, x, y))
\]
where $\epsilon_{inst}$ is the transfer efficiency of the instrument
(depending on factors such as the filters used, optical coatings etc.),
$Q$ is the normalised and `centred' point spread function, and $g$ gives
the position of the centre of the PSF.

Finally, we need to know how the detector responds to incident radiation;
the `detector response' map,
\[
P_{count} = P_{count} (s_0, [t_0, t_1], E_{det}(t))
\]
where $P_{count}$ is the probability distribution over output count images,
and depends on the integration time $[t_0, t_1]$, the state $s_0$ of the detector
at time $t_0$, and the detector irradiance as a function of time, $E_{det}(t)$.
This relationship has been stated very generally, to account for the possibility
of e.g.\ fluctuating sky state, fluctuating detector state, correlated pixels, etc.
However, to a fairly good approximation, we expect detector pixels to be linear
and uniform in their response to the integrated irradiance over their surfaces,
and for separate pixels to have approximately independent responses. Then,
we have
$P_{count} \approx \prod_{p \in pixels} P_p$
where $P_p$ is the probability distribution for the output count of pixel $p$.
This should satisfy
\[
\mathbb{E}(E_p) \approx r_p \left (\int_{t_0}^{t_1} dt \int_A dp_x\, dp_y\, E(t, p_x, p_y)\right)
\]
where $r_p$ is response efficiency (in appropriate units) of pixel $p$. Of course,
this discussion has neglected the quantum nature of light, but for large photon
fluxes the continuous approximation should be good, with quantum discrete-ness
providing some of the probability distribution's spread (and the rest due to instrumental
noise etc.).

TODO : talk about cosmic rays?

This report is concerned with MOSFIRE's spectroscopic mode, so we are interested
in the properties of the instrument when the CSU is in use. Assuming non-overlapping
slits, we can split the detector irradiance as
\[
E_{det}(p_x, p_y, L_{sky}) = \sum_{s \in slits} \int d\lambda\, dx\, dy\, L_{sky}(\lambda, x, y) t_s[\lambda, x, y](p_x, p_y)
\]
i.e. with a separate transfer function for each slit, which will have
transfer efficiency $\epsilon_s (\lambda, x, y)$ equal to zero outside
the small $x,y$ domain viewed by the slit. Furthermore, assuming that
slits are thin in the $x$ direction and comparatively long in $y$, then
if $L$ and $g$ and $Q$ do not vary significantly over the relevant $x$
range (say $x_0 \pm \Delta x / 2$), we have
\[
E_{det}(p_x, p_y, L_{sky}) \approx
\sum_{s \in slits} \int d\lambda\, dy\, \Delta x\, L_{sky}(\lambda, x_0(y), y)
E_s(\lambda, x_0(y), y) Q_s[\lambda, x_0(y), y]((p_x, p_y) - g_s (\lambda, x_0(y), y))
\]
where $E_s(\lambda, x_0(y), y)$ is the transfer efficiency integrated
over $x$. The effect, if $Q_s$ is fairly concentrated, is that the slit
$s$ defines an effective mapping $\lambda, y \mapsto g_s (\lambda,
x_0(y), y) = p_x, p_y$. That is, light entering a slit from a specific
(`vertical') sky position and at a specific wavelength is mapped to
a specific position on the detector. In fact, the operation of the
instrument is such that the map approximately separates into an affine
$y \mapsto p_y$ transform, and an approximately exponential $\lambda
\mapsto p_x$ transform caused by the reflection grating (see REF,
especially figure 2). However, this approximation is not quite accurate
enough for our purposes, and in the sections below we generally
consider the $p_x, p_y \leftrightarrow \lambda, y$ relationship for a given
slit (at a given position).

Need to say anything more here?
