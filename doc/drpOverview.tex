\section{DRP Overview}

As described in~\cite{surfreport}, the purpose of the DRP is to go
from raw instrument outputs to object spectra. Naively, if we had an
accurate $p_x, p_y \mapsto \lambda$ map for a slit, and the raw slit
image was $f(p_x, p_y) \propto \lambda(p_x, p_y)$, i.e.\ proportional to the target
object's spectral power, then $\lambda \circ f$ would give a set of
samples from the object's spectrum. Of course, the raw image has many
other factors affecting it, all of which need to be compensated for in
some way --- these include
%
\begin{itemize}
\item Detector response varying between pixels.
\item Instrument transfer function varying with $y$ and $\lambda$
\item Atmospheric transparency varying with time and wavelength
\item Atmospheric emission (night sky lines) contributing to image
\end{itemize}
%
along with others. Unfortunately, there wasn't time during my SURF project
to tackle these other areas properly --- I thought about all of them to some degree,
but didn't get to the stage of proper prototype code. As a result, the prototype
DRP currently consists only of the `calibration' stages; those involved in
determining the $p_x, p_y \leftrightarrow \lambda, y$ relationship.

This task has been factored in three main stages
\begin{itemize}
\item Determining the $p_x, p_y \mapsto y$ map, as discussed in
section~\ref{sec:edgeTracing}.
\item Determining the $p_x, p_y \mapsto \lambda$ map, up to a monotonic
function of $\lambda$; referred to as `relative wavelength calibration'
(section~\ref{sec:relativeWavelengthCalibration}).
\item Determining the $p_x, p_y \mapsto \lambda$ map absolutely;
referred to as `absolute wavelength calibration' (section~\ref{sec:absoluteWavelengthCalibration}).
\end{itemize}
