\section{Inferring profiles}
\label{sec:profileInference}

There are a number of situations in which we might have an approximately
separable image, i.e.\ $f(p_x, p_y) \approx f_1(p_x)f_2(p_y)$, and want
to infer one or both of $f_1$, $f_2$. An example might be in performing
reduction of a 2D spectrum to 1D --- if the object's spatial profile can
be estimated accurately, then we can use this to lower the effective
noise level of the extracted spectrum, by weighting down the wings of
the profile, which will be mostly noise, and weighting up the centre,
where the strongest signal will be.

A simple approach to fitting a fairly flexible such `profile function'
was implemented, but proved to be a be a bit too simple, not coping well
with noise (basically, to infer the $p_y$ profile, it `guessed' the
$p_x$ profile via the total magnitude of each column, then used this
guessed profile to form a weighted sum --- with too much noise, the
guessed $p_x$ profile was bad enough to cause problems). Further thought
on how to solve this problem, and what exactly is necessary here,
is needed.
