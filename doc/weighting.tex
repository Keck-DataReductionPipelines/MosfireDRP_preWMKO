\section{Weighting observations / error tracking}
\label{sec:weighting}

Since the DRP is supposed to produce scientifically-useful information,
it is clearly important to estimate the uncertainty in its outputs,
and this entails tracking such uncertainty throughout the computation.
Presently, we use pretty much the simplest possible such scheme ---
every data vector carries round a corresponding weight vector of inverse
variances. That is, if $v_i$ is the data vector, then we are modelling
it as being the sum of a `true' data vector $u_i$ and an error vector
$\epsilon_i$, where the weight vector $w_i$ is such that $\epsilon_i
\sim N(0, 1/w_i)$ (with the $\epsilon_i$ assumed independent).
This is clearly a fairly crude way of doing things, but is better
than nothing --- it also enables a unified treatment of bad pixels etc.,
which can simply have their inverse variances set to zero, corresponding
to their conveying no information.

In the current implementations of the simulator and DRP, the simulated
readout noise is the same for each pixels, so there is no difference in
the initial per-pixel weights. However, if we want to compensate for
effect of differing pixel QEs (by dividing each observed pixel value by
the known pixel QE), then we need to modify the weight attached to the
new value accordingly (i.e.\ the weight is multiplied by the square of
the pixel QE).

TODO : discussing of handling photon counting `noise' in this framework.

TODO : more discussion here?
